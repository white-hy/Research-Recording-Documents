\documentclass[twocolumn]{article}
%\usepackage[UTF8]{ctex}
\usepackage{amsmath}
\usepackage{longtable}
\usepackage{booktabs}
\usepackage{graphics}
\usepackage{float}
\title{Weekly Report}
\author{Haozhe Feng}
\begin{document}
	\maketitle{}
\section{The tasks I have finished this week}
This week I mainly focus on the implementation of Faster R-CNN and Mask R-CNN. I have completely understood every single detail of Faster R-CNN, done a report about the technique difficulties of implementation and draw a flow chart. I find that the source code of Faster R-CNN has been highly industrialized and is of value reading, so I plan to study from the source code and make it the foundation of my model.\\

\section{The tasks I plan to do next week}
Next week I will still focus on the implementation and hope to transfer the model into our medical image dataset, which no one has done before.\\

\vspace*{.3cm}
\noindent And recently Stanford begins a course about the mathematical theory of deep learning, referring to cohomology theory, PDE , Probablistic Theory(Random Matrix and Kernel Based Methods for Hypothesis Testing) for deep learning, on which I want to spend some time. I also notice that the paper about mathematic theory for deep learning has increased a lot on arxiv, and many have been submitted to ICLR (such as Bindong's PDE-Net\cite{Dong},he uses PDE theory to design a precise network which only uses half deepth and 1/3 parameters of Resnet but achieve the same performance as ResNet)
\bibliography{Report}
\bibliographystyle{unsrt}
\end{document}