\documentclass[8pt]{beamer}
\usepackage{ctex}
\usepackage{graphics}
\usepackage{amsmath}
\usepackage{xcolor} 
\usepackage{float}
\usefonttheme[onlymath]{serif}
\usetheme{Warsaw}
%\usecolortheme{dolphin}
\setbeamercovered{transparent}
%%-------------------------------------------------
\title{Weekly Report}
\author{冯浩哲}

\begin{document}
	\frame{\titlepage}
		
	%%-------------------------------------------------
	
	\section*{Outline}	

\begin{frame}[fragile]
\frametitle{Outline}
\begin{itemize}  
	
\item The tasks I have finished this week
\vspace{.5cm} 
\item Things we want to do next week
\vspace{.5cm}
\end{itemize}
\end{frame}

\section*{The main tasks I have finished this week}	
\begin{frame}[fragile]
\frametitle{The tasks I have finished this week}
This week, I mainly focus on 2 articles\cite{46351},\cite{Wang2017}.The first one is an important article illustrate the details of new deep learning structure "Capsules with dynamic routes" written by Hinton and the last one is a new article about deep learning in lung nodule segmentation. \\
\vspace{.5cm}
I write the literature study reports for the 2 articles, especially for Hinton's paper. However, Hinton's paper show a number of state-of-the-art ideals, and I can't fully understand all of them, so there are some doubts and uncertainty in my study report.
\vspace{.3cm}
\end{frame}


\section*{Things we plan to do next week}	
\begin{frame}[fragile]
\frametitle{ Things we plan to do next week}
We will still do our research and practice some ideas according to the summary of  discussion with Professor Chen last week, and we will record our problems and discuss them with Professor Chen next Monday.
\end{frame}

\bibliography{10_29}
\bibliographystyle{plain}
\end{document}